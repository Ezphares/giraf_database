\section{Requirements Analysis}
It is a proven fact that miscommunication between developers and customers or users can lead to misunderstood, unnecessary or unwanted functionality.\cite{req_wrong} %Larman (2004)
As a result, development time is wasted on functionality that will not be used. The goal with the requirements analysis is to end up with a list of concrete requirements that will satisfy the customer's demands and will fulfil the problem statement without wasting time on unnecessary tasks. 

The requirements have been collected from some of the contacts mentioned in the common report and from Ulrik Nyman, the semester coordinator. The Wasteland project is a bit peculiar in this context, because it handles the data behind the functionality in \ac{giraf} and not so much of the functionality itself. The reason for Ulrik Nyman's inclusion is, that this is a student project and someone else will take over later on. And Ulrik Nyman will be involved in future development of \ac{giraf} and he has some specific requirements in this context.

\subsection{Contact Group Requirements}
The contact persons held a lecture about how they use the various tools and techniques available to them and which pros and cons they each had. They have been involved in previous versions of \ac{giraf} and have tested some of the existing functionality such as Wombat (see \autoref{cha:common}). They mentioned that some of their most useful tools were things such as timers that visualize the time spent and time remaining. They were interested in being able to take pictures with the tablets and assigning the pictures to one or several children in the department. They also requested that children were not dependent on a specific tablet and that one tablet could accommodate several children with their own specific preferences. The contact group mentioned that they sometimes take the children on excursions where there rarely is any Internet connection. They said that it would be nice if the tablets could be used in such a setting and the changes that were made would then only be saved locally until they got a Wi-Fi connection e.g. pictures taken on the excursion.

Ulrik Nyman had a few but more specific requirements. Such as install instructions that were simple and easy to understand and follow. He also specified that he preferred a small amount of well written and appropriately commented code over a large amount of uncommented and less structured code. Lastly he asked for some good documentation of the work done by each project group along with a list of functionality that would be well suited in the future. 

\subsection{Specific Applications}
\label{sub:specific_applications}
The various applications in the \ac{giraf} system are divided into tools and games. Common for all applications is that they need to be able to distinguish between the different profiles. The games, for example, may need to be able to save settings and high scores for the individual child and some tools need to be able to access the child's pictograms. In these cases the applications need rights to add data to the database.  
% subsection specific_applications (end)

\subsection{Admin Group} 
\label{sub:admin_group}
The Admin group is responsible for all desktop administration of the \ac{giraf} system. Their goal is to allow administrators desktop access to the various children's profiles and allow them to create, edit and delete profiles. They also allow for administrators to make changes that affect entire departments, thus making it a lot easier to add some specific pictograms to all children of a given department.   

\subsection{Security}
\label{sub:security}
As of the summer 2013 the \ac{giraf} project is still in development. The current login system consists of a QR-code that the guardians scan with the tablet's camera in order to log in. The QR-code contains the username and password for each individual guardian and the QR-codes can easily be copied and could present a security breach. When the system eventually gets released to the public, there needs to be a high degree of security implemented in \ac{giraf}. The system will eventually contain personal and sensitive information and there are laws that regulate how this information should be managed.  

\subsection{List of Requirements}
\label{sub:list_of_requirements}
The activities mentioned in the previous sections have been evaluated and the requirements established from these are listed below:

\begin{itemize}
	\item A central database
	\item A local database on each tablet		
	\item Synchronization between the central and the local databases
	\item Android application that can control the synchronization
	\item An \ac{api} that provides applications with easy database access
	\item Ability to use local database without Internet access.
	\item An install manual
	\item Good documentation
\end{itemize}

The requirements are fairly straightforward, however the last three requirements are a bit different from the rest. The ability to use the local database without Internet access is not essential, but would be nice if there is time as specified by one of the contact persons. The two last requirements are Ulrik Nyman's requirements as there have previously been problems with poorly documented code and install instructions. This is a very important requirement as the \ac{giraf} system will be further developed, either by students or professionals.