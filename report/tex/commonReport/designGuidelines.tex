\subsection{Design Guidelines}
\label{sub:designGuidelines}
The purpose with the guidelines is to get a consistent look and feel across all of the different applications included in the \ac{giraf} system.
The design guidelines have been discussed among all of the project groups, and they are as follows:

\begin{quote}
\begin{itemize}
\item Keep the existing color palette
\item Font: Helvetica
\item Font size: use common sense. \emph{Android} offers extra small/small/medium/large/huge
\item Minimize the use of text, use images instead of text
\item \ac{gui} in vector graphics
\item Green and red are universal colors for `accept'/`cancel'
\item Applications have animal icons
\item Icons are non-customizable
\item Every application should be locked in landscape mode
\end{itemize}
\end{quote}

The color palette will be the same as in the 2012 version of \ac{giraf}. With regards to font type and size, Helvetica has been chosen and developers need to keep in mind, that the text has to be readable on the tablet.

The aim is to use more images and less text as the target audience are mostly children, many of which have communication and/or reading difficulties and some have problems imagining objects purely from text.

The \ac{gui} will be in vector graphics, because it scales well, which makes it possible to reuse some of the images. Green and red are universal colors for `accept'/`cancel'. It may sound obvious but other applications have been developed with different colors.
Tool-applications should have animal icons.

Lastly everything will be in landscape mode as this eliminates additional implementation for responsive layout, when the tablet is rotated.
