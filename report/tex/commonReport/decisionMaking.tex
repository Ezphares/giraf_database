\section{Decision Making - The Process}
\label{sec:decisionmaking}

The following section will describe the decision making process, set in place to ensure that everyone would be heard on an equal and democratic footing. The decision making process during this semester's multi-project consists of two different steps.

\subsection{The Weekly Meeting}
\label{sub:theweeklymeeting}

It was strongly recommended by the semester coordinator, Ulrik Nyman, to hold a weekly meeting for all software students on the bachelor semester of 2013. The meeting's agenda consists of a few points of formalism at the very beginning, in which a secretary and a moderator are chosen by means of voting. Candidates for these roles are entirely self-appointing and a vote is issued to pick one of the candidates. %and as such resolves any potential issues of personal discomfort (peer pressure or otherwise). A vote is then issued to pick one of the candidates.

Though the weekly meeting is established to ensure a higher level of communication between students, as well as ensure that decisions will be taken on a multi-project level scale, not all points are actually discussed at this meeting. Instead, a committee approach is agreed upon, see \secref{sub:committees}. The purpose of establishing committees is to ensure that relevant discussions to a given topic can be had, but within a smaller audience.

Committees are discussed at the weekly meeting where voting determines which committees are established. A chairman for a committee is self-appointed and a vote determines if there is consent to let the given person be chairman.

The meeting will then proceed and discuss the ideas and suggestions agreed upon within each committee from the previous week and at the multi-project level determine, by voting, which ideas are okay, or if any of the points concluded by one of the committees are subpar and should be reworked.

\subsection{Rules of Conduct}
\label{sub:rulesofconduct}
During the first weekly meeting some general rules of conduct were established, including decisions on how voting should be done. A number of ways to do this were suggested. Ultimately it was decided that every person present at the meeting has an individual vote, and the idea of a group based voting system was therefore discarded. Furthermore in the event that there is a 50/50 split, the vote will have to be reissued. There must be majority 'for' or 'against' a decision. Guidelines for when a decision should be taken at the weekly meeting were established as well. If a decision involved only two or three groups, then it would not be necessary to discuss at the weekly meeting. If, however, the decision impacted everyone, a committee would be established to make these decisions.

%Det er her jeg beskriver det med at modsætte sig en komit beslutning
%------------
%Decisions agreed upon in committees can be overruled by those attending the weekly meeting - however to disagree with a decision taken in a committee one has to write a letter for everyone to see explaining one's arguments. This letter must be accessible prior to attending a meeting so that everyone has a chance to read these arguments. %\todo{Blev det besluttet ? eller det mig der har d{\aa}rlig hukommelse?}
%------------

During a committee meeting every group has a single vote. It is possible to send as many group members as is deemed necessary to the committee meetings, however, it does not increase the number of total votes a group has. %% Every group should keep in mind that the entire point of having committees is to reduce the amount of people attending a discussion significantly.

\subsection{Committees}
\label{sub:committees}
%A committee is established when the decision to be made within it requires consent from all or most groups, and it is deemed a too deep topic or challenging discussion to be discussed at a weekly meeting. This is typically because the decision(s) to be made, impacts everyone to some degree.
A committee ideally consists of a representative from each multi-project group and a chairman agreed upon at the weekly meeting. The chairman is responsible for setting up the meeting, time, place, agenda as well as writing down the details of what is agreed upon during the committee meeting. %The group representatives are responsible for attaining knowledge about the topic to be discussed to give a more educated opinion about the topic.
%This ensures that the weekly meeting stays an 'information-sharing' focused meeting whereas the committees will be in charge of making the tough decisions, that does not require consensus from every person participating in the multiproject.

The resulting work product of the committee is a document, that potentially answers every question on the agenda, ready to be presented at the next multi-project meeting.

\subsubsection*{Important Committees}
\label{subsub:importantcommittees}
The following section describes an extract of some of the most important committees, that were established during one of the first weekly meetings.

\begin{itemize}
        \item Wiki: \emph{Ensures that the multi-project wiki page on Redmine is created in a uniform way by establishing guidelines for new
        articles.}
        \item Design Guidelines: \emph{Ensures that the User Interface design of the \ac{giraf} application is uniform (e.g. in regards to font, color scheme and various buttons - green for `yes' and red for `no').}
        \item Common Report: \emph{This committee is responsible for the creation of the common-report chapters, which you are reading now, that are at the beginning of every semester report.}
        \item Pictogram Class: \emph{Because every group requires a common pictogram class, it was decided to create a Pictogram Class committee to determine the functionality that this class needed.}
        \item GIT: \emph{The GIT committee is responsible for working out a common structure across all repositories to create uniformity and make it easier to continuously integrate.}
        \item Public Pictogram: \emph{Determines guidelines for how pictograms are handled in the database (e.g. who has access rights to what and why?).}
        \item Story: \emph{The story committee is responsible for creating a story to follow every sprint. It puts the sprint's tasks into an overall context.}
        \item CI/Git: \emph{This committee is responsible for coming up with solutions to potential issues that might occur as part of the Continuous Integration step when using GIT.}
\end{itemize}
