\section{Server API}

When several different applications needs access to data from the database, it is important to have a flexible \ac{api}, especially when 
the applications needs access for very different reasons, e.g. administration, games and pictures. Considering that several people have to
work with it, and that new groups will be using the \ac{api} later, simplicity is also a primary concern. Several decisions were made, all of
which is to support the simplicity and flexibility of the \ac{api}.

\subsection{Philosophy}

Considering that the \ac{api} should allow for all the necessary operations on a persistent storage system, it was decided that is should
implement the \ac{crud} actions, as these provide the minimal number of operations that allow for complete data manipulation. \citep{crud13}
As such, the \ac{crud} actions comply with both demands of simplicity and flexibility.\p
Another decision was made to keep the \ac{api} design free of authorization elements. It is, for example, completely valid, for a normal user
to request deletion of every profile in the database, and then it is up to the server to determine that they do not have the right to do this.
This, again, keeps things simple, as it means that the same actions are available, whether the request comes from an administrative or
user context.

Additionally, it does not create additional authentication work for the server implementations, since checking whether a user has a given right
should be checked each time in any case.

\subsection{Data Serialization Format}

When communicating on a network, it is important that the data can be serialized, that is, represented as a series of symbols. Two of the most
well-known standards for serializing data between different platforms were considered, namely \ac{json} and \ac{xml}. The main difference, from
an overall system point of view, between these, is that \ac{json} is more light-weight, and arguably easier for humans to read, but has less 
ways of expressing data than \ac{xml}.

This choice was discussed with multiple groups and in the end, \ac{json} was selected. The readability of the format was valued highly, and it
was argued that \ac{json} would have enough power of expression for what this \ac{api} would need. The fact that it is less verbose, and as such
transferred faster over networks is also welcome.

\subsection{Request Structure}

For simplicity, the decision was made that requests of every type should have the same structure. That is, the request should be in the form of
a \ac{json} object with three keys: \texttt{auth}, \texttt{action} and \texttt{data}. The \texttt{auth} key should contain another object, which
contains the authentication information of the following request, whether logging in with a username and password, or using a QR certificate. The
\texttt{action} key refers to a string, naming one of the \ac{crud} actions. Finally, the \texttt{data} key, refers to an object which contents
will differ depending on the action named. An example of requesting each profile that a user can access, can be seen in \autoref{lst:request.sample}.

\lstinputlisting[label=lst:request.sample,caption=Sample Request]{lst/request.sample.json}

\subsection{Response Structure}

As with the requests, responses should have a single structure. Incidentally, the response is also a \ac{json} object containing three keys, in this
case \texttt{status}, \texttt{errors} and \texttt{data}. The \texttt{status} key should contain a single string, describing what happened during the
processing of the request, this could for example be ``OK'', ``SYNTAXERROR'', or ``AUTHFAILED''. The \texttt{errors} key refers to an array of error
messages, typically an empty array if everything went fine. The \texttt{data} key can either refer to null, if an update or delete request was issued,
or if there was errors in the request, a list of new ids if it was a create request, and finally the requested data for a read request. An example of
a possible response to the request seen in \autoref{lst:request.sample} can be seen in \autoref{lst:response.sample}

\lstinputlisting[label=lst:response.sample,caption=Sample Response]{lst/response.sample.json}

\subsection{Overview}

The final \ac{api} is simple, but powerful enough that it can do what it needs to do, partly due to the adherence to the \ac{crud} principles. The full
documentation for the \ac{api} can be seen in %TODO appendix.
