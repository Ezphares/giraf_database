\section{Product Backlog}
The tasks were prioritized and added to sprint backlogs and as work progressed new task were added to the backlog and tasks that were completed got a check mark. Due to the fact that estimation in software development is often very difficult, the estimates were done in numbers representing the expected effort required to complete them. The priority goes from 1-10, with 1 being highest priority. The following table contains the product backlog for the project.

\begin{center}
	\begin{tabular}{| c | p{4cm} | c | c | l |}
	\hline
	\textbf{ID} 	& \textbf{Name} 										& \textbf{Priority} 	& \textbf{Estimated time} 	& \textbf{Status} 			\\ 	\hline
	1 	& Connection Library 						& 3 		& 5 				& Finished. 		\\	\hline
	2 	& Database Library 							& 4 		& 3 				& Finished. 		\\	\hline
	3 	& Authentication 							& 2 		& 5 				& In progress. 		\\ 	\hline
	4 	& Database schema 							& 1 		& 2 				& Finished. 		\\	\hline
	5 	& Design database API 						& 2 		& 5 				& Finished. 		\\	\hline
	6 	& JSON encoder/decoder 						& 2 		& 2 				& Finished. 		\\	\hline
	7 	& OasisLib dummy							& 5 		& 3 				& Finished. 		\\	\hline
	8 	& Implement API calls						& 1 		& 21 				& Known bugs. 		\\ 		\hline
	9 	& Establish SQLite database 				& 2 		& 3 				& Finished. 		\\	\hline
	10	& Implement synchronization					& 3 		& 8 				& Finished. 		\\	\hline
	11	& Write report								& 1 		& 13 				& Finished. 		\\	\hline
	12	& Proofread report 							& 4 		& 8 				& Finished. 		\\ 	\hline
	13	& Install instructions 						& 4 		& 2 				& Finished. 		\\	\hline
	\end{tabular}
\end{center}