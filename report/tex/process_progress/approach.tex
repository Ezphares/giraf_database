\section{Approach}
As described in the common chapter, a multi-project of this nature requires a lot of structure and planning. In one of the first multi-group meetings, all of the group's agreed on a development method, namely SCRUM. \todo{reference common chapter} But the method that the individual group's was up to the individual group's to decide. The decision was that all of the groups met on a weekly basis and discussed the progress of the individual groups and problems that might arise.  

\subsection{SCRUM implementation} % (fold)
\label{sub:scrum_implementation}
For simplicity the group decided to use the same development method that was used on the multi-project level. There were however some alterations made to better suit the group's development style.

\subsubsection{Sprint Lengths}
Lectures during the semester meant that work was divided into half days.\todo{reference common chapter} The group decided to use the sprint lengths agreed upon at the common meetings. 

\subsubsection{Test-Driven Development}
The group wanted to try some of XP's techniques as we found them interesting and we had some positive experience with them from previous semesters. In this case we wanted to try an XP approach where the tests are written before the actual code that will make the test succeed.

\subsubsection{Pair Programming}
Pair Programming has been borrowed from the XP-development method as the group had positive experience with the technique and we found that it is well suited to programming in a learning environment. 
% subsection scrum_implementation (end)