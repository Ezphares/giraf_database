\section{Builder Functions}
The so-called builder functions are implemented to facilitate ease of reading and writing in the \ac{api} calls.

\subsection{Fix}
For all read calls to the \ac{api}, some data from the database has to be altered to conform to the \ac{api} standard either by renaming or fixing the type of the data. In order for this to happen, two functions are implemented,  \lstinline|fix_rename| and  \lstinline|fix_type|, which as their names imply rename and change the type respectively. 

\subsection{Extractors}
The \ac{api} needs to extract data from \ac{json} objects in order to build the queries to the database, and to prevent misuse of the queries some of these data need to be escaped and all need to be validated. This is done with three extractor functions: 
\begin{itemize}
\item \lstinline|extract_string| 
\item \lstinline|extract_int| 
\item \lstinline|extract_bool|. 
\end{itemize}
In \autoref{lst:extract_string} the \lstinline|extract_string| function is shown.

\lstinputlisting[label=lst:extract_string, caption=Extract String, language=c++]{lst/extract_string.cpp}

Each of the functions are similar in that they all validate that the given key is of the expected type (in \autoref{lst:extract_string} this would be string) and if not they return an error. Apart from validating the type, \lstinline|extract_string| also escapes the string using the database module, if a database is provided in the input.

\subsection{Builders}
These functions create a data type from another - e.g. building a \ac{json} object from a row from the result of a database query, building an array from the result of a query, or build a string to fit with the SQL \lstinline|IN| keyword. The builder functions are:
\begin{itemize}
\item \lstinline|build_object_from_row|
\item \lstinline|build_array_from_query|
\item \lstinline|build_simple_array_from_query|
\item \lstinline|build_in_string|
\item \lstinline|build_simple_int_vector_from_query|
\item \lstinline|build_simple_ind_map_from_query|
\end{itemize}

In \autoref{lst:build_object_from_row} the code for the \lstinline|build_object_from_row| function is shown. 

\lstinputlisting[label=lst:build_object_from_row, caption=Build Object from Row, language=c++]{lst/build_object_from_row.cpp}

This function is called when one wants to create a \ac{json} object from the result of a query. It takes a row (the result of a query) and a fixture, which is a function pointer to a function which applies one or more of the fixes mentioned earlier. 

The function loops through the row and directly translates it to a \ac{json} object.

\lstinputlisting[label=lst:build_array_from_query, caption=Build Array from Query, language=c++]{lst/build_array_from_query.cpp}

\autoref{lst:build_array_from_query} utilizes the database module to fetch a row from the result, and  while there are more rows, calls the \lstinline|build_object_from_row| and appends the \ac{json} objects to a \ac{json} array.

\subsection{Validators}
As the name implies, these functions validate that a given value (or set of values) exist in an array or a vector. The validator functions are:
\begin{itemize}
\item \lstinline|validate_array_vector|
\item \lstinline|validate_value_in_vector|
\end{itemize}

\lstinline|validate_array_vector| takes an array and a vector and loops through to determine if every value in the array exists in the vector. 

\lstinline|validate_value_in_vector| takes a single value and checks, if this value exists in the vector.

Both functions return boolean values of either true or false, depending on whether the value(s) were found or not.