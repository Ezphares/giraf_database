\section{Background}
The objective for this semester's project was to continue the work on the Giraf project that has been handed over by last year's bachelor students. 

As mentioned in the \todo{insert reference to common chapter}common chapter the Giraf project is \todo{make sure that the common chapter includes a clear description}broken up into several subprojects. 

We chose to work on the database part. Initially the task was to enable synchronization between the central database and the local databases on the individual tablets.

But unfortunately the latest version of the last years database project seemed to be missing, because the only version that we could find was incomplete. This meant that our workload had increased substantially as we now had to implement our own central and local database as well as handle the synchronization between the two.

There were of course some positives to take from the initial set back. We now had the chance to model both databases to our liking, giving us full control. We started by reevaluating the database schema, using the last years groups schema as a mock-up and editing where necessary. This saved us some time as a lot of the schema could be reused.

Last year's server was written in Java and the argumentation was that the students had a lot of experience using Java.
We on the other hand have had introductory courses in C and C\# and have little to none experience using Java. Another argument was that a server has to be able to handle a large amount of requests. The fact that we are writing a server side application and the fact that C performs better than Java, we feel more comfortable writing our server in C instead of Java.
