As mentioned in \autoref{cha:common}, \ac{giraf}'s main objective is to improve communication between children
with autism and their guardians. The primary tool for the communication will be a tablet that the children will
use either alone or with their guardian. In this context a single tablet needs to be able to accommodate many
different children with individual preferences. The children should not be dependent on one particular tablet, but
should be able to use any tablet with the \ac{giraf} system installed. 
As a consequence the \ac{giraf} project will need two types of databases. A local database on the individual
tablets and a central database that the tablets should be able to synchronize with. As children with autism have
very specific needs, their individual preferences need to be stored on one tablet and possibly retrieved on
another. Without the databases the \ac{giraf} system will not be able to store individual preferences and will not
be able to synchronize preferences across different tablets.

A problem statement will be given at the end of this chapter.