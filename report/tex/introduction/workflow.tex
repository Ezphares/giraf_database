\section{Workflow}
This section will describe the act of accessing the central database through API's, and explain the steps we take for this to happen.

When an application is requesting a file from the local database in OasisLib and the file is not in the local database, the request will be sent to our server, which makes a request to the central database instead. The request from OasisLib will be written in JSON, and will be decoded by the server such that the central database can understand it. The response from the central database will be sent back to OasisLib, encoded in JSON. This way, the application does not have to know whether the requested data is in the local or central database, the application will get the data either way, if there is a steady connection to the server.

Each application has in the database a table that is associated with the profile of a child, this table includes a settings blob with settings for that specific child. This settings blob is entirely controlled by the application, which means that the application itself must encode and decode the blob.

The administration interface is different than the regular Android applications because it can connect directly to the central server and edit settings and profiles in the central database. This is done because it is needed to be able to manage users and profiles for the children, as well as managing their rights.